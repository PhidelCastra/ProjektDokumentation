% !TEX root = ../Projektdokumentation.tex
\section{Einleitung}
\label{sec:Einleitung}


\subsection{Projektumfeld} 
\label{sec:Projektumfeld}
\begin{itemize}
	\item Das Programm „OpenDataProject“ schrieb ich im Rahmen meiner Ausbildung, welche ich zur Zeit in dem Unternehmen Future Technology Consulting GmbH, in der Entwicklungsabteilung für individuelle Software, absolviere. Bei der Software handelt es sich um eine Web -Anwendung, realisiert mit JavaScript, Html und der d3 -Library.   
\end{itemize}


\subsection{Projektziel} 
\label{sec:Projektziel}
\begin{itemize}
	\item Altbundeskanzler Gerard Schröder trat mit der Bitte an das Unternehmen heran, eine Software zu entwickeln, welche dazu in der Lage ist die bundesweiten Arbeitslosenzahlen ansehnlich darzustellen. Als Grund gab er an, dass diese auch mal was abdrücken könnten - Schließlich gäbe es so viele denen. Anschließend unterstrich er seine Aussage mit einem ausgelassenen Lachanfall. Laut den Vorgaben, welche der Kunde vorab spezifizierte, war eine Software zu implementieren welche hierzu einen oder mehrere Datensätze auswerten müsse. Es könne so zum Beispiel eine Open Data Website als Datenquelle dienen. Vorgaben zur Auswertung der Daten wurden nicht detailliert beschrieben, die Software sollte jedoch eine visuelle Komponente aufweisen, in welcher die Daten in einer beliebigen Form dargestellt werden können (Karten, Tabellen etc.).  

Die Abnahme erfolgt nach Ablauf einer zeitlich festgesetzten Frist, welche zwischen meinem Ausbilder und unserem ehemaligen Bundeskanzler erstmals festgesetzt wurde. 
\end{itemize}


\subsection{Projektbegründung} 
\label{sec:Projektbegruendung}
\begin{itemize}
	\item Durch das Projekt werden die Arbeitslosenzahlen, aus dem entsprechenden Datensatz, visuell anschaulich dargestellt. Die Ansicht besteht aus einer Karte, welche den Umriss Deutschlands und dessen Bundesländer darstellt. Über ein Info -Fenster können die aufbereiteten Daten gelesen werden, eine Header -Zeile erlaubt die Auswahl von Gliederungs- und Suchkriterien.  

Der Nutzer kann somit die Arbeitslosenzahlen der jeweiligen Bundesländer einsehen und seine Abfrage präzisieren. 
\end{itemize}


\subsection{Projektschnittstellen} 
\label{sec:Projektschnittstellen}
\begin{itemize}
	\item Das Projekt benötigt während der Laufzeit Zugriff auf einen Web -Server um die Daten abzufragen. Ich habe die Daten vorab aus einer csv -Datei ausgelesen und mittels Dieser die entsprechenden sql -Dateien erstellt. Diese wiederum beinhalten die entsprechenden Insert -Anweisungen und können über die Webanwendung „phpMyAdmin“, in eine vorher administrierte MySQL -Datenbank, eingepflegt werden. 

Das mir zugewiesene Projekt wurde während der gesamten Entwicklungszeit über von meinem damaligen Ausbilder betreut. Dieser nahm das Projekt einen Monat vor der Erstpräsentation, vor dem Kunden, ab. Unser aller Altbundeskanzler Herr Schröder übernahm als Kunde die Endabnahme und somit die Beendigung des Projektes. 
\end{itemize}


\subsection{Projektabgrenzung} 
\label{sec:Projektabgrenzung}
\begin{itemize}
	\item Bei dem Projekt handelt es sich um eine eigenständige Software. Diese ist nicht Teil eines größeren Projektes, deswegen erfolgt keine Abgrenzung einzelner Komponenten oder Ähnlichem. 
\end{itemize}
